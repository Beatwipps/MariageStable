
\title{Les Mariages Stables}
\author{
        Jordann Perrotta \\
        Aix-Marseille Universit\'e\\
}
\date{\today}

\documentclass[11pt]{article}
\usepackage[frenchb]{babel}
\usepackage[Algorithme]{algorithm}
\usepackage{algorithmic}
\begin{document}
\maketitle

\tableofcontents

\newpage
\section{Introduction}
En 1962, David Gale et Lloyd Shapley ont pr\'esent\'e le probl\`eme du mariage
stable dans un article intitul\'e "Les admissions aux universit\'e et la
stabilit\'e du mariage".
Ils ont pos\'e et r\'epondu \`a la question de savoir s'il est possible de
trouver un mariage stable impliquant $n$ hommes et $n$ femmes.
Dans leur r\'eponse, ils pr\'esentent l'algorithme suivant pour
trouver un mariage stable.
L'algorithme Gale-Shapley implique un certain nombre de "rounds" (ou
"it\'erations"). Dans le premier tour, chaque homme non engag\'e propose 
\`a la femme qu'il pr\'ef\`ere le plus, puis chaque femme 
r\'epond «peut-être» au pr\'etendant qu'elle pr\'ef\`ere et «non» \`a tous
les autres pr\'etendants. Elle est alors provisoirement «engag\'ee» pour le
pr\'etendant qu'elle pr\'ef\`ere jusqu'ici, et ce pr\'etendant lui est
\'egalement engag\'e provisoirement.
Dans chaque tour suivant, chaque homme non engag\'e propose \`a la
femme qu'il pr\'ef\`ere le plus parmis les femmes auquel il n'a pas encore
propos\'e (peu importe si la femme est d\'ej\`a engag\'ee),puis chaque femme r\'epond «peut-être»
si elle est actuellement pas engag\'e ou si elle pr\'ef\`ere cet homme plutot
que son partenaire actuel (dans ce cas, elle rejette son partenaire provisoire actuel
qui devient non engag\'e).
Le caractère provisoire des engagements conserve le droit d'une femme d\'ej\`a
engag\'ee de «n\'egocier» (et, en l'occurrence, de «jeter» son partenaire
jusqu'\`a ce moment-l\`a).
Ce processus est r\'ep\'et\'e jusqu'\`a ce que tout le monde soit engag\'e.

%\`A la fin de cet algorithme, nous sommes garantis que tout le monde est
%mari\'es.
%En fin de compte, il ne peut y avoir un homme et une femme \`a la fois non
%engages, car il doit lui avoir propos\'e \`a un moment donne (puisqu'un homme
%finira par proposer \`a tout le monde, le cas \'ech\'eant) et, s'il est
%propos\'e, elle serait n\'ecessairement engag\'ee (\`a quelqu'un) par la suite.
%Puis, cet algorithme nous garantit que le mariage est stable. Soit $f$ une
% femme et $h$ un homme. Ils sont tout engag\'es par des partenaire diff\'erents. \`A la
%fin de l'algorithme, il n'est pas possible que $f$ et $h$ pr\'ef\`erent 
%l'un l'autre sur leurs partenaires actuels.
%Si $h$ pr\'ef\`ere $f$ \`a son partenaire actuel, il a du propos\'e \`a
%$f$ avant de proposer \`a son partenaire actuel. Si $f$ a accept\'e sa
%proposition, mais qu'elle n'est pas mari\'ee avec lui \`a la fin, c'est qu'elle
%du l'abandonn\'e pour quelqu'un qu'elle pr\'ef\'erait plus, et n'a donc pas
%envie de $h$ plus que son partenaire actuel.
%Si $f$ a rejet\'e sa proposition, elle \'etait d\'eja avec quelqu'un qu'elle
%aimait plus que $h$.
\subparagraph{Proposition 1}
L'algorithme de Gale-Shapley se termine.
\subparagraph{}
Soit $n$ hommes et $n$ femmes impliqu\'ees dans l'algorithme. Donc, un homme
doit proposer \`a au plus $n$ femme avant d'\^etre accept\'e ou rejet\'e par
cette derni\`ere.
Donc au plus $n^2$ propositions peuvent se produire, apr\`es quoi l'algorithme
se termine.

\subparagraph{Proposition 2}
\`A la fin de cet algorithme, tout le monde est mari\'es.
\subparagraph{}
Supposons la contradiction que $m$ est un homme non mari\'e \`a la fin de
l'algorithme Gale-Shapley. Ensuite, il y a forc\'ement une femme libre $w$,
car il y a le m\^eme nombre d'hommes et de femmes et personne ne peut \^etre
mari\'e \`a plus d'une personne. Donc, si une femme obtient une
proposition, elle sera mari\'ee lorsque l'algorithme se terminera.
Donc, $w$ n'a re\c{c}u aucune proposition. Mais, afin que l'algorithme puisse
se terminer, l'homme doit \^etre mari\'e, ce qu'il n'est pas le cas,
ou a \'et\'e rejet\'e par chaque femme, y compris $w$. Donc $m$ a d\^u
propos\'e \`a $w$, ce qui est une contradiction. Donc $m$ doit \^etre mari\'e
\`a la fin de l'algorithme. Il s'ensuit immédiatement que chaque femme doit se
marier \`a la fin de l'algorithme de Gale-Shapley.

\subparagraph{Th\'eor\`eme}
L'algorithme de Gale-Shapley produit un mariage stable.
\subparagraph{}
Supposons une contradiction selon laquelle l'algorithme de Gale-Shapley produit
une correspondance instable pour une instance du probl\`eme du mariage stable. 
Donc, il existe une paire $(m, w')$, de sorte que $m$ pr\'ef\`ere $w'$ \`a $w$, 
son partenaire assign\'e, et $w'$ pr\'ef\`ere $m$ \`a $m'$, 
son partenaire assign\'e. Ensuite, $m$ a propos\'e \`a $w'$ avant qu'il ait
propos\'e \`a $w$, puisque $w'$ est avant $w$ sur sa liste. Mais une femme ne
peut que rejeter un homme si elle re\c{c}oit une proposition d'un homme qu'elle
pr\'ef\`ere.
Donc, si une femme rejette un homme, c'est qu'elle pr\'ef\`ere son dernier
partenaire \`a l'homme rejet\'e.
Donc $w'$ pr\'ef\`ere $m'$ \`a $m$, ce qui est une contradiction. 
Ainsi, l'algorithme de Gale-Shapley produit une correspondance stable.

\section{D\'efinition}

Donnons nous pour ces definitions un ensemble d'hommes $H$ et un ensemble
de femmes $F$

\subsection{Couplage}
Un couplage $c$ est une relation fonctionnelle et injective entre $H$ et $F$.

En d’autres termes, c'est une bijection entre un sous-ensemble de $H$ et un
sous-ensemble de $F$.

\subsection{Couplage parfait}
Un couplage est parfait si c’est une bijection entre $H$ et $F$.


\subsection{Couplage unstable}
Un couplage est unstable si il existe un homme $h$ et une femme $f$ qui ne sont
pas partenaire mais chacun d'entre eux prefere le partenaire de l'autre dans ce
couplage. On dit que cette paire $h$ - $f$ est une paire bloquante.

\subsection{Couplage stable}
Un couplage est stable si il n'y a pas de paire bloquante. De mani\`ere plus
formel, un couplage est stable si il n'existe aucun homme $h$ et aucune femme
$f$ qui serait tenter d'interchanger de partenaire.
De plus un couplage est stable si un homme $h$ prefere une autre femme $f$ mais
cette derniere ne prefere pas $h$ \`a son partenaire actuel et vice-versa.

\subsection{Ordre total}
Un ordre total est l'ordre de pr\'ef\'erence de chaque personne.
Pour chaque $h \in H$, on a donc un ordre total $\leq_h$ sur $F$, et pour
chaque $f \in F$ , un ordre total $\leq_f$ sur $H$.
On note $f_1$ $\leq_h$ $f_2$ si $h$ pr\'ef\`ere $f_1$ \`a $f_2$.
On note $f_1 \leq_h f_2$ si $h$ pr\'ef\`ere $f_1$ \`a $f_2$.

\section{Algorithmes et impl\'ementations}\label{algo}
Dans cette section, les algorithmes sont impl\'ement\'es en Java ainsi que tout
les structures de donn\'es. Pour commencer, afin de faciliter le choix des
diff\'erents algorithme pour la r\'esolution d'une instance de mariage stable,
le patern design "Strat\'egie" semble \^etre un bon choix. La strat\'egie de
r\'esolution va d\'ependre de plusieurs facteurs, soit du type de donn\'e (si
la gestion des indiff\'erence doit \^tre prise en consid\'eration), soit du
choix de l'utilisateur. Si l'utilisateur d\'ecide d'utiliser une strat\'egie
qui n'est pas optimal pour le type de donn\'e choisie, le programme va lui
indiquer le meilleur algorithme pour la r\'esolution du probl\`eme. Chaque
strat\'egie a une impl\'ementation diff\'erente, il y en a 3, une pour
l'algorithme basique de Gale-Shapley, une pour l'algorithme weakly stable de
Robert W.Irving et une autre pour le Strongly stable du m\^eme auteur.
\subsection{Basic Stable}

\subsubsection{Donn\'es}
\subsubsection{PseudoCode}
\begin{algorithm}
\caption{Basic Stable}
\begin{algorithmic} 
\REQUIRE Initialiser tout les $m \in M$ et $w \in W$ a libre
\ENSURE Un couplage stable
\WHILE {∃ un homme libre $m$ qui peut encore proposer a une femme $w$}
\STATE $w \leftarrow$ la premiere femme dans la liste de $m$ a qui $m$ n'a pas
encore propose
\IF{$w$ est $libre$}
\STATE $(m, w)$ devient engage
\ELSE [il existe deja un couple $(m', w)$]
\IF {$w$ prefere $m$ a $m'$}
\STATE $m'$ devient libre
\STATE $(m, w)$ s'engage
\ELSE
\STATE $(m', w)$ se reengage
\ENDIF
\ENDIF
\ENDWHILE
\end{algorithmic}
\end{algorithm}
    
\subsection{Weakly Stable}
\subsubsection{Donn\'es}
\subsubsection{PseudoCode}
\begin{algorithm}
\caption{Basic Stable}
\begin{algorithmic} 
\REQUIRE Initialiser tout les $m \in M$ et $w \in W$ a libre
\ENSURE Un couplage stable
\WHILE {∃ un homme libre $m$}
\STATE $w \leftarrow$ la premiere femme dans la liste de $m$
\STATE $m$ propose et devient engage a $w$
\IF{$w$ est $libre$}
\STATE $(m, w)$ devient engage
\ELSE [il existe deja un couple $(m', w)$]
\IF {$w$ prefere $m$ a $m'$}
\STATE $m'$ devient libre
\STATE $(m, w)$ s'engager
\ELSE
\STATE $(m', w)$ se reengage
\ENDIF
\ENDIF
\ENDWHILE
\end{algorithmic}
\end{algorithm}

    
\subsection{Strong Stable}
\subsubsection{Donn\'es}
\subsubsection{PseudoCode}

\section{Results}\label{results}
In this section we describe the resul

\section{Conclusions}\label{conclusions}
We worked hard, and achieved very little.

\bibliographystyle{abbrv}
\bibliography{main}

\end{document}
This is never printed