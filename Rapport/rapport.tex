	
\title{Les Mariages Stables}
\author{
        Jordann Perrotta \\
        Aix-Marseille Universit\'e\\
}
\date{\today}

\documentclass[11pt]{article}
\usepackage[utf8]{inputenc}
\usepackage[T1]{fontenc}
\usepackage[frenchb]{babel}
\usepackage[Algorithme]{algorithm}
\usepackage{algorithmic}
%%% francisation des algorithmes
\renewcommand{\algorithmicrequire} {\textbf{\textsc{Entrées:}}}
\renewcommand{\algorithmicensure}  {\textbf{\textsc{Sorties:}}}
\renewcommand{\algorithmicwhile}   {\textbf{tantque}}
\renewcommand{\algorithmicdo}      {\textbf{faire}}
\renewcommand{\algorithmicendwhile}{\textbf{fin tantque}}
\renewcommand{\algorithmicend}     {\textbf{fin}}
\renewcommand{\algorithmicif}      {\textbf{si}}
\renewcommand{\algorithmicendif}   {\textbf{finsi}}
\renewcommand{\algorithmicelse}    {\textbf{sinon}}
\renewcommand{\algorithmicthen}    {\textbf{alors}}
\renewcommand{\algorithmicfor}     {\textbf{pour}}
\renewcommand{\algorithmicforall}  {\textbf{pour tout}}
\renewcommand{\algorithmicdo}      {\textbf{faire}}
\renewcommand{\algorithmicendfor}  {\textbf{fin pour}}
\renewcommand{\algorithmicloop}    {\textbf{boucler}}
\renewcommand{\algorithmicendloop} {\textbf{fin boucle}}
\renewcommand{\algorithmicrepeat}  {\textbf{répéter}}
\renewcommand{\algorithmicuntil}   {\textbf{jusqu'à}}

\floatname{algorithm}{Algorithme}

\let\mylistof\listof
\renewcommand\listof[2]{\mylistof{algorithm}{Liste des algorithmes}}

% pour palier au problème de niveau des algos
\makeatletter
\providecommand*{\toclevel@algorithm}{0}
\makeatother

%\listofalgorithms % pour lister les algos (après la toc)
\begin{document}
\maketitle

\tableofcontents

\newpage
\section{Introduction}
Les algorithmes pour trouver des solutions au problème du mariage stable ont des applications dans diverses situations du monde réel, peut-être la plus connue étant celle des étudiants diplômés en médecine à leurs premiers offres d'emploi. Une application importante et à grande échelle d'un mariage stable consiste à affecter les utilisateurs aux serveurs d'un grand service Internet distribué. Des milliards d'utilisateurs accèdent à des pages Web, des vidéos et d'autres services sur Internet, ce qui oblige chaque utilisateur à correspondre à l'un des centaines de milliers de serveurs du monde entier qui offrent ce service. Un utilisateur préfère des serveurs suffisamment proches pour fournir un temps de réponse plus rapide pour le service demandé, ce qui entraîne une commande préférentielle des serveurs pour chaque utilisateur. Chaque serveur préfère servir les utilisateurs à un coût inférieur, ce qui entraîne une commande préférentielle des utilisateurs pour chaque serveur. Les réseaux de distribution de contenu qui distribuent la plupart du contenu et des services du monde résolvent ce problème de mariage stable complexe entre utilisateurs et serveurs toutes les secondes afin de permettre à des milliers d'utilisateurs d'être associés à leurs serveurs respectifs qui peuvent fournir les pages Web, les vidéos demandées, et tant d'autres services.
\\Dans la Section 2, nous aborderons les définitions relatives au mariage stable ainsi que plusieurs exemples d'illustrations. Dans la section 3, nous expliquerons les différents types de recherche de  mariage stable qu'il existe, leurs pseudo-codes, quelques proposition et théorèmes, ainsi que certaines qualités de mesure d'évaluation du mariage stable que l'on peut obtenir. Dans la section 4, nous détaillerons les algorithmes de résolution du  mariage stable dans le langage Java, les structures de données utilisés ainsi que leurs utilisations. Pour finir, dans la section 5, nous verrons comment utiliser ce projet, son installation et les issus possibles de résultat.

%Ils ont posé et répondu à la question de savoir s'il est possible de
%trouver un mariage stable impliquant $n$ hommes et $n$ femmes.
%Dans leur réponse, ils présentent l'algorithme suivant pour
%trouver un mariage stable.
%L'algorithme Gale-Shapley implique un certain nombre de "rounds" (ou
%"itérations"). Dans le premier tour, chaque homme non engagé propose 
%à la femme qu'il préfère le plus, puis chaque femme 
%répond "peut-être" au prétendant qu'elle préfère et "non" à tous
%les autres prétendants. Elle est alors provisoirement "engagée" pour le
%prétendant qu'elle préfère jusqu'ici, et ce prétendant lui est
%également engagé provisoirement.
%Dans chaque tour suivant, chaque homme non engagé propose à la
%femme qu'il préfère le plus parmi les femmes auquel il n'a pas encore
%proposé (peu importe si la femme est déjà engagée),puis chaque femme répond "peut-être"
%si elle est actuellement pas engagé ou si elle préfère cet homme plutot
%que son partenaire actuel (dans ce cas, elle rejette son partenaire provisoire actuel
%qui devient non engagé).
%Le caractère provisoire des engagements conserve le droit d'une femme déjà
%engagée de "négocier" (et, en l'occurrence, de "jeter" son partenaire
%jusqu'à ce moment-là).
%Ce processus est répété jusqu'à ce que tout le monde soit engagé.

%à la fin de cet algorithme, nous sommes garantis que tout le monde est
%mariés.
%En fin de compte, il ne peut y avoir un homme et une femme à la fois non
%engages, car il doit lui avoir proposé à un moment donne (puisqu'un homme
%finira par proposer à tout le monde, le cas échéant) et, s'il est
%proposé, elle serait nécessairement engagée (à quelqu'un) par la suite.
%Puis, cet algorithme nous garantit que le mariage est stable. Soit $f$ une
% femme et $h$ un homme. Ils sont tout engagés par des partenaire différents. à la
%fin de l'algorithme, il n'est pas possible que $f$ et $h$ préfèrent 
%l'un l'autre sur leurs partenaires actuels.
%Si $h$ préfère $f$ à son partenaire actuel, il a du proposé à
%$f$ avant de proposer à son partenaire actuel. Si $f$ a accepté sa
%proposition, mais qu'elle n'est pas mariée avec lui à la fin, c'est qu'elle
%du l'abandonné pour quelqu'un qu'elle préférait plus, et n'a donc pas
%envie de $h$ plus que son partenaire actuel.
%Si $f$ a rejeté sa proposition, elle était déja avec quelqu'un qu'elle
%aimait plus que $h$.
%\subparagraph{Proposition 1}
%L'algorithme de Gale-Shapley se termine.
%\subparagraph{}
%Soit $n$ hommes et $n$ femmes impliquées dans l'algorithme. Donc, un homme
%doit proposer à au plus $n$ femme avant d'être accepté ou rejeté par
%cette dernière.
%Donc au plus $n^2$ propositions peuvent se produire, après quoi l'algorithme
%se termine.
%
%\subparagraph{Proposition 2}
%à la fin de cet algorithme, tout le monde est mariés.
%\subparagraph{}
%Supposons la contradiction que $m$ est un homme non marié à la fin de
%l'algorithme Gale-Shapley. Ensuite, il y a forcément une femme libre $w$,
%car il y a le même nombre d'hommes et de femmes et personne ne peut être
%marié à plus d'une personne. Donc, si une femme obtient une
%proposition, elle sera mariée lorsque l'algorithme se terminera.
%Donc, $w$ n'a reçu aucune proposition. Mais, afin que l'algorithme puisse
%se terminer, l'homme doit être marié, ce qu'il n'est pas le cas,
%ou a été rejeté par chaque femme, y compris $w$. Donc $m$ a d\^u
%proposé à $w$, ce qui est une contradiction. Donc $m$ doit être marié
%à la fin de l'algorithme. Il s'ensuit immédiatement que chaque femme doit se
%marier à la fin de l'algorithme de Gale-Shapley.
%
%\subparagraph{Théorème}
%L'algorithme de Gale-Shapley produit un mariage stable.
%\subparagraph{}
%Supposons une contradiction selon laquelle l'algorithme de Gale-Shapley produit
%une correspondance instable pour une instance du problème du mariage stable. 
%Donc, il existe une paire $(m, w')$, de sorte que $m$ préfère $w'$ à $w$, 
%son partenaire assigné, et $w'$ préfère $m$ à $m'$, 
%son partenaire assigné. Ensuite, $m$ a proposé à $w'$ avant qu'il ait
%proposé à $w$, puisque $w'$ est avant $w$ sur sa liste. Mais une femme ne
%peut que rejeter un homme si elle reçoit une proposition d'un homme qu'elle
%préfère.
%Donc, si une femme rejette un homme, c'est qu'elle préfère son dernier
%partenaire à l'homme rejeté.
%Donc $w'$ préfère $m'$ à $m$, ce qui est une contradiction. 
%Ainsi, l'algorithme de Gale-Shapley produit une correspondance stable.
\newpage
\section{Définition}

%En 1962, David Gale et Lloyd Shapley ont présenté le problème du mariage stable dans un article %intitulé "Les admissions aux université et la stabilité du mariage". 
Donnons nous pour ces définitions un ensemble d'hommes $H$ et un ensemble
de femmes $F$ de même cardinalité.

\subsection{Couplage}

Un couplage $c$ est une relation fonctionnelle et injective entre $H$ et $F$.

En d’autres termes, c'est une bijection entre un sous-ensemble de $H$ et un
sous-ensemble de $F$.
Chaque homme peut être marié à au plus une femme, et vice-versa.

\subsection{Couplage parfait}
Un couplage est parfait si c’est une bijection entre $H$ et $F$.
En d'autre terme, aucun homme et aucun femme ne reste célibataire.\\\\

Pour la suite de ces définitions, on suppose que chaque homme classe toutes les femmes par ordre de
préférence, et inversement.

\subsection{Couplage stable}
Un couplage est dit stable si pour tout couple $(h, f)$ et $(h', f')$, il n'existe pas une interprétation tel que  $h$ préfère $f'$ à son partenaire actuel et $f'$ préfère $h$ à son partenaire actuel. De la même manière, la stabilité d'un couplage est possible dans le cas ou $h$ préfère $f'$ à son partenaire actuel, par contre $w'$ ne préfère pas $h$ à son partenaire actuel. 
%Il y a deux cas de figure pour un couplage stable:
%\subparagraph{}
%- Si il n'existe aucun homme $h_1$ et aucune femme
%$f_2$ qui serait tenter d’inter-changer de partenaire, le couplage est stable.
%\subparagraph{}
%- Si un homme $h_1$ préfère une autre femme $f_2$ mais $f_2$ ne préfère pas $h_1$ à son partenaire actuel, le mariage est stable. Idem dans l'autre sens, c'est à dire si une femme $f_1$ préfère une autre homme $h_2$ mais $h_2$ ne préfère pas $f_1$ à son partenaire actuel. 

\subsection{Ordre}
Un ordre, dans le cas des mariages stables, est la place qu'occupe un homme $h$ dans les préférences d'une femme $f$ et vice-versa. Chaque homme (respectivement femme) doit ordonner de manière décroissante ses préférences. Il existe deux type d'ordre, l'ordre partiel et l'ordre total.

\subsection{Ordre partiel}
Un ordre partiel, est un ordre qui admet une indifférence de personne à une position donnée dans les préférences. Pour l'exemple, un homme $h_1$, en position numéro 1, préférera autant une femme $f_1$ quune autre femme $f_2$.


\subsection{Ordre total}
Un ordre total est un ordre qui n'admet aucune indifférence de personne à une position donnée dans les préférences.
Pour chaque $h \in H$, on a donc un ordre total $\leq_h$ sur $F$, et pour
chaque $f \in F$ , un ordre total $\leq_f$ sur $H$.
On note $f_1$ $\leq_h$ $f_2$ si $h$ préfère $f_1$ à $f_2$.
On note $f_1 \leq_h f_2$ si $h$ préfère $f_1$ à $f_2$.

\section{Algorithmes et résolution}
Il existe 2 types de résolution pour le problème du mariage stable.
Le premier type est l'algorithme basique de Gale/Shapley, permettant de résoudre le problème de mariage stable lorsque les préférences des hommes et des femmes sont ordonnées sans ex-æquo possible. Le deuxième type se divise en sous-type, l'algorithme de Gale-Shapley étendu, la version dite "SUPER" (que nous n'aborderons pas dans ce rapport), puis la version dite "STRONG". Ce deuxième type existe dans le sens ou, il est possible d'avoir une indifférence de préférence entre tel et tel personne. Dans la vrai vie, il est tout à fait possible d'avoir des préférences ex-æquo, choisir entre une boulangerie ou une autre, un coiffeur ou un autre, un supermarché ou un autre, etc.. La version basique de résolution est un cas particulier qui peux s'appliquer uniquement si les choix réalisés sont strictement ordonnés. Ces type de résolution sont en réalité un sorte de mesure, si nous obtenons un couplage stable avec le "weakly", il est possible qu'un certain homme $h$ peut préférer une autre femme $w'$ que sa partenaire actuel, alors que $w'$ ne préférera moins $h$ à son partenaire actuel. De la même manière, nous pouvons affirmer que si un couplage stable à été trouvé par l'algorithme "STRONG", il n'existera aucun couple $(h, f)$ et $(h', f')$ tel que  $h$ préfère $f'$ à son partenaire actuel et $w'$ préfère $h$ à son partenaire actuel. Dans la suite de cette section, nous allons voir les pseudo-codes de chaque type de résolution.
\newpage
\subsection{Basique}
On fait évoluer un couplage partiel (i.e., pas parfait), initialement vide
\subsubsection{Pseudo-Code}
\begin{algorithm}
\caption{Basic Stable}
\begin{algorithmic} 
\REQUIRE 
\begin{list}{}{'}
	\item $H \leftarrow$ ensemble des hommes
	\item $F \leftarrow$ ensemble des femmes
    \item $liste_{pref}(f) \leftarrow$ liste de préférence de f
    \item $liste_{pref}(h) \leftarrow$ liste de préférence de h
\end{list}      

\STATE Initialiser tout les $h \in H$ et $f \in F$ à libre
\WHILE {il existe un homme libre $m$ qui peut encore proposer à une femme $f$}
\STATE $f \leftarrow$ la première femme dans la liste de $h$ a qui $h$ n'a pas
encore propose
\IF{$f$ est $libre$}
\STATE $(h, f)$ devient engagé
\ELSE [il existe déjà un couple $(h', f)$]
\IF {$f$ préfère $h$ a $h'$}
\STATE $h'$ devient libre
\STATE $(h, f)$ s'engage
\ENDIF
\ENDIF
\ENDWHILE 
\ENSURE Un couplage stable
\end{algorithmic}
\end{algorithm}

\subsection{Weakly}
\subsubsection{Pseudo-Code}
\begin{algorithm}
\caption{Weakly Stable}
\begin{algorithmic} 
\REQUIRE 
\begin{list}{}{'}
	\item $H \leftarrow$ ensemble des hommes
	\item $F \leftarrow$ ensemble des femmes
    \item $liste_{pref}(f) \leftarrow$ liste de préférence de f
    \item $liste_{pref}(h) \leftarrow$ liste de préférence de h
\end{list} 

\STATE Initialiser tout les $h \in H$ et $f \in F$ à libre
\WHILE {il existe un homme libre $m$ qui peut encore proposer à une femme $f$}
\STATE $f \leftarrow$ la première femme dans la liste de $h$ a qui $h$ n'a pas
encore propose
\STATE $h$ propose, et devient engagé à $f$
\IF {il existe un homme $h'$ qui est déjà engagé avec $f$}
\STATE assigner $h'$ à libre
\ENDIF
\FORALL {successeur $h''$ de m  dans $liste_{pref}(f)$	}
\STATE supprimer la paire $(h'', f)$
\ENDFOR
\ENDWHILE
\ENSURE Un couplage stable
\end{algorithmic}
\end{algorithm}
\subsubsection{Explication}

Cet algorithme revient à utiliser l'algorithme basique, auquel cas nous avons arbitrairement ordonné les indifférences. 
\subsection{Strong}
\subsubsection{Pseudo-Code}
\begin{algorithm}
\caption{Strong Stable}
\begin{algorithmic} 
\REQUIRE 
\begin{list}{}{'}
	\item $H \leftarrow$ ensemble des hommes
	\item $F \leftarrow$ ensemble des femmes
    \item $liste_{pref}(f) \leftarrow$ liste de préférence de f
    \item $liste_{pref}(h) \leftarrow$ liste de préférence de h
\end{list} 
\STATE Initialiser tout les $h \in H$ et $f \in F$ à libre
\REPEAT 
	\WHILE  {il existe un homme libre $h$ qui peut encore proposer à une femme $f$}
		\FORALL {femme à la tête de $liste_{pref}(h)$}
			\STATE $h$ propose, et devient engagé à $f$
			\FOR {chaque successeur strict $h'$ de $h$  dans $liste_{pref}(f)$}
				\IF {$h'$ est engagé à w}
					\STATE briser l'engagement
				\ENDIF
				\STATE supprimer la paire $(h',f)$
			\ENDFOR
		\ENDFOR
	\ENDWHILE
	\IF{ les couples obtenus ne forment pas un couplage parfait}
		\STATE trouver la zone critique $Z$ des hommes
		\FORALL {femme $f$  qui est engagé à un homme dans $Z$}
			\STATE briser tout les engagements impliquant $f$
			\FORALL{homme $h$ à la fin de $liste_{pref}(f)$}
				\STATE supprimer la paire $(h, f)$				
			\ENDFOR
		\ENDFOR
	\ENDIF
\UNTIL{ce que chaque $liste_{pref}(h)$ soit vide ou que tout le monde soit engagé}
\IF {tout le monde est engagé}
	\STATE les engagements forment un couplage fortement stable 
\ELSE
\STATE il n'existe pas de couplage fortement stable 
\ENDIF

\end{algorithmic}
\end{algorithm}

\section{Algorithmes et implémentations}\label{algo}
Dans cette section, les algorithmes sont implémentés en Java ainsi que tout
les structures de donnés. Pour commencer, afin de faciliter le choix des
différents algorithme pour la résolution d'une instance de mariage stable,
le patern design "Stratégie" semble être un bon choix. La stratégie de
résolution va dépendre de plusieurs facteurs, soit du type de donné (si
la gestion des indifférence doit être prise en considération), soit du
choix de l'utilisateur. Si l'utilisateur décide d'utiliser une stratégie
qui n'est pas optimal pour le type de donné choisie, le programme va lui
indiquer. Chaque stratégie a une implémentation différente, il y en a 3, une pour
l'algorithme basique de Gale-Shapley, la seconde est celle du "WEAKLY" et la dernière, l'algorithme "STRONG". Nous allons donc diviser cette section en 3 parties.
\subsection{Les Structures de données}
Le projet se compose de plusieurs fichiers: 

\begin{list}{•}{}
\item \emph{StableStategie.java}, une interface.
\item \emph{LireFichier.java}, une classe qui permet de lire un fichier texte contenant les preferences et de les affecter dans les $Maps$ correspondantes.
\item \emph{BasicStable.java}, la classe qui implémente l'algorithme basique.
\item \emph{WeaklyStable.java}, la classe qui implémente l'algorithme $weakly$.
\item \emph{StrongStable.java}, la classe qui implémente l'algorithme $strong$.
\item \emph{Instance.java}, une classe qui permet de créer une instance en fonction de la stratégie, des deux $Maps A$ et $B$ (des préférences) ainsi que des listes des personnes participants au couplage.
\item \emph{Main.java}, la classe principal qui exécute le programme.
\item \emph{Convertable.java}, une petite interface qui permet de convertir une $Map$ de différente manière, selon la stratégie utilisé.
\end{list}

Dans l'ordre des choses, l'interface \emph{StableStategie} contient la méthode pour trouver un couplage, selon la stratégie utilisé, l’implémentation est différente, c'est pour cela que l'on a besoin d'une classe qui va créer cette instance avec les différents paramètres requis. La classe \emph{Instance} va donc créer un instance en lui précisant la stratégie ainsi que les maps des deux groupes à coupler puis de leur nom. Pour ce qui est des paramètres des personnes, c'est la classe \emph{LireFichier} qui s'en charge. Le constructeur prend en paramètre le fichier texte correspondant au préférence d'un groupe de personne, ainsi que la strategie requise. La lecture du fichier texte va affecter dans une $Map$ les personnes ainsi que chacune de leurs préférences. Chaque case de la map sera donc de type $A \rightarrow [B, C, D, ...]$ pour le cas de la stratégie basique, pour celle qui admet les indifférences la map sera de type $A \rightarrow [[B], [C], [D], ...]$ avec la possibilité d'avoir à une position donnée, plusieurs personnes. L'utilisation d'une map Java me semble être la manière la plus évidente pour traiter ce genre de sujet, chaque clé représente une personne, et chaque valeur représente les préférences, l’accès aux données est simplifié. Une fois avoir créer cette instance, l’exécution de la stratégie se fait par la méthode \emph{executeStableStrategie()} qui va dépendre de la stratégie utilisé. C'est une bonne manière de déléguer le travail afin d'avoir une bonne cohésion du code et une bonne lisibilité, ce qui permet de deboger facilement si jamais quelque chose ne s'est pas bien déroulé. Pour la classe Main, c'est ici que se déroule le programme. en lançant l'application, l'utilisateur sera guidé tout au long de l'avancement du programme, il pourra faire des choix qui lui amèneront à tel ou tel résultat. L'utilisateur aura la possibilité de mesurer la qualité de son résultat, ainsi de savoir si le résultat obtenu est satisfaisant ou pas. L'utilisateur pourra lancer le programme X fois afin d'avoir une moyenne, ou de comparer un résultat à un autre et de ne garder que le meilleur. 



\subsection{Implémentation basique}
\subsection{Implémentation Weakly}
\subsection{Implémentation Strong}



\section{Installation}

Très simple, une fois le .jar téléchargé, dans un terminal :\\
java -jar nomDuJAR.jar

\section{Conclusion}\label{conclusions}
Ce projet informatique a été enrichissant, j'ai pu approfondir mes connaissances en matière de programmation, j'ai appris a utilisé Github afin de rendre ce projet accessible a mes tuteurs et aux autre utilisateurs, et d’être dans une optique un peu plus professionnelle du travail. Ce sujet est très intéressant car il est utile dans la vie courante, je suis content d'avoir pu participer a l'élaboration de ce projet. Je voudrai remercier mes tuteurs Petru Valicov et Kolja Knauer pour toute l'aide qu'ils ont pu m'apporter, tout les conseils et tout les échange par mail, qui m'ont grandement aider pour ce projet.

\bibliographystyle{abbrv}
\bibliography{main}

\end{document}
This is never printed
